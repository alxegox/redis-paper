\chapter{Deployment}

Redis is a senior key-value database. It is similar to memcached, but the data persistence, and the support of the data type is very rich. Redis can be viewed as a data structure of server.
Redis all data are stored in the memory, then do not regularly by the asynchronous mode is saved to disk (this is called "semi persistent mode"); can also take every data changes are written to an append only file (AOF). (this is called "persistence model").
\begin{lstlisting}[
caption=Installation,
basicstyle={\ttfamily},
keywordstyle={\color{blue}\ttfamily},
stringstyle={\color{red}\ttfamily},
commentstyle={\color{green}\ttfamily},
breaklines=true
]
1. Download address:
$ wget http://redis.googlecode.com/files/redis-2.6.13.tar.gz
2. Decompression
$ tar xzf redis-2.6.13.tar.gz
3. Compile
$ cd redis-2.6.13
$ make
$make install
$cp redis.conf /etc/
Parameter introduction:
The make install command execution is completed, will generate the executable file in the /usr/local/bin directory, respectively is redis-server, redis-cli, redis-benchmark, redis-check-aof, redis-check-dump, following their role:
redis-server: Redis server daemon startup programs
redis-cli: The Redis command line tools. You can also use the telnet protocol to operate according to the text
redis-benchmark: Redis performance testing tool, test the Redis under the present system read and write performance
redis-check-aof: Data repair
redis-check-dump: Check the export tool
\end{lstlisting}
\begin{lstlisting}[
caption=Modify and start,
basicstyle={\ttfamily},
keywordstyle={\color{blue}\ttfamily},
stringstyle={\color{red}\ttfamily},
commentstyle={\color{green}\ttfamily},
breaklines=true
]
4. Modify the system configuration file, execute the command
a) echo vm.overcommit_memory=1 >> /etc/sysctl.conf
b) Sysctl vm.overcommit_memory=1 or echo vm.overcommit_memory=1>>/proc/sys/vm/overcommit_memory
The use of digital means:
0, Said the kernel will check to see if you have enough memory available for process used; if there is enough free memory, memory application allows; otherwise, the application memory failure, and the error is returned to the application process.
1, Kernel allows allocation of physical memory for all, regardless of the current memory state.
2, Kernel allows allocating more than all the physical memory and swap space is the sum of the memory
5. Modify the redis configuration file
a) $ cd redis-2.6.13
b) vi redis.conf
c) Daemonize yes--- to modify the process running in the background
Parameter introduction:
daemonize: If after daemon operation
pidfile: PID file location
port: The port number.
timeout: Request timeout
loglevel: Log information level
logfile: Log file location
databases: The number of open databases
save * *: Save the snapshot of the frequency, the first * said how long, third * indicates how many times the write operation. Write a number in a certain period of time, automatically save the snapshot. Can set up multiple conditions.
rdbcompression: Whether the use of compression
dbfilename: A snapshot of data file name (just the file name, not including the directory)
dir: A snapshot of data storage directory (this is the directory)
appendonly: Whether to open the appendonlylog, open the words each write operation will take a log, it will improve the data anti risk ability, but influence the efficiency.
appendfsync: Appendonlylog how to synchronize to the disk (three options, respectively is every write all forced to call fsync, enable a fsync per second, do not call the fsync wait for the system to its own synchronization)
6. Start redis
a) $ cd /usr/local/bin
b) ./redis-server /etc/redis.conf
7. Check whether the successful start
a) $ ps -ef | grep redis
\end{lstlisting}
