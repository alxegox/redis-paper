\chapter{Partitioning}

Splitting Redis into multiple instances is a process called partitioning \cite{redis_partitioning}. Every instance will only contain a small subset of the keys.

Partitioning servers two main goals:
\begin{itemize}
	\item The Database can be much larger because using multiple computers merged into one is more powerful than using a single instance
	\item Scaling becomes reality. Giving control over how much cores and instances used in your cluster is a must for bigger systems
\end{itemize}

\section{Types}
There are several ways to split up your keys and route your users to what they need.

\subsection{Range Partitioning}
As the name suggests you define ranges of object into specific Redis instances. This system requires a table that maps ranges to instances. It needs to be managed and it's needed for every kind of object. Because of that it's much more inefficient than other methods.
\subsection{Hash Partitioning}
An alternative is hash partitioning. This method just needs a key. The process is as follows:
\begin{itemize}
	\item Hash the key to turn it into a number e.g. with the crc32 hash function
	\item Using modulo on this number and turn it into a number between 0 and n-1. Where n is my number of available Redis instances
\end{itemize}

\section{Where and how implemented?}
The responsibility can be part of different parts of a software stack
\begin{itemize}
	\item Client side partitioning pushes the responsibility choosing a fitting node to the client
	\item Proxy assisted partitioning adds a proxy layer between the client and a Redis node. It chooses the right one for you
	\item Query routing allows sending to any instance and choosing the right one for you
\end{itemize}

\section{Disadvantages}
Everything can't be gold so there has to be some problems.
\begin{itemize}
	\item Operations that need multiple sets on different instances are not possible
	\item Transactions involving multiple keys cannot be used
	\item Backup and logging (RDB and AOF) becomes more complex because it has to done for every instance
\end{itemize}